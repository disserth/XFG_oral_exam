\documentclass{beamer}

\usepackage{graphicx}
\usepackage[latin1]{inputenc}
\usepackage[T1]{fontenc}
\usepackage[english]{babel}
\usepackage{listings}
\usepackage{xcolor}
\usepackage{eso-pic}
\usepackage{mathrsfs}
\usepackage{url}
\usepackage{amssymb}
\usepackage{amsmath}
\usepackage{multirow}
\usepackage{booktabs}
\usepackage{hyperref}
\usepackage{booktabs}
\usepackage{bbm}
\usepackage{cooltooltips}
\usepackage{colordef}
\usepackage{beamerdefs}
\usepackage{lvblisting}
\usepackage{color}

\pgfdeclareimage[height=3.5cm]{logobig}{hucaselogo}
\pgfdeclareimage[height=0.7cm]{logosmall}{Figures/LOB_Logo}

\renewcommand{\titlescale}{1.0}
\renewcommand{\titlescale}{1.0}
\renewcommand{\leftcol}{0.6}

\title[lmplied Binomial Tree ]{Put Option Price}
\authora{Yinan Wu}
\authorb{}
\authorc{}

\def\linka{http://lvb.wiwi.hu-berlin.de}
\def\linkb{http://case.hu-berlin.de}
\def\linkc{}

\institute{Ladislaus von Bortkiewicz Chair of Statistics \\
C.A.S.E. -- Center for Applied Statistics\\
and Economics\\
Humboldt--Universit\"at zu Berlin \\}

\hypersetup{pdfpagemode=FullScreen}

\begin{document}
\frame[plain]{
\titlepage
}
\frame{
\begin{itemize}
\item Arrow-Dedreu prices $\lambda_n^i$ (discount risk-neutral probability)\\
the price of an option that pays 1 in one and only one state i at $n$th level, and otherwise pays 0.\\
\begin{equation*}
\begin{aligned}
\lambda_{n+1}^1&=e^{-r\Delta t}\{(1-p_n^1)\lambda_n^1\}\\
\lambda_{n+1}^{i+1}&=e^{-r\Delta t}\{\lambda_n^ip_n^i+\lambda_n^{i+1}(1-p_n^{i+1})\}\\
\lambda_{n+1}^{n+1}&=e^{-r\Delta t}\{\lambda_n^np_n^n\}
\end{aligned}
\end{equation*}
\bigskip
\item{put option price $P(K,n\Delta t)$}
$
P(K,n\Delta t)=\sum_{i=0}^n\lambda_{n+1}^{i+1}\max(K-S_{n+1}^{i+1},0)
$
\end{itemize}
}
\frame{
with $K=S=S_n^i$ and $S_n^{i+1}<S_n^{i}<S_{n+1}^{i+1} $ we have:
\begin{equation*}
\begin{aligned}
P(S,n\Delta t) &=e^{-r\Delta t}[\lambda_n^1(1-p_n^1) \max(S-S_{n+1}^1,0)\\
&+ \sum_{j=1}^{n-1}\{\lambda_n^jp_n^j+\lambda_n^{j+1}(1-p_n^{j+1})\} \max(S-S_{n+1}^{j+1},0)\\
&+\lambda_n^np_n^n \max(S-S_{n+1}^{n+1},0)]\\
&=e^{-r\Delta t}[\lambda_n^1(1-p_n^1)(S-S_{n+1}^1)\\
&+\sum_{j=1}^{i-1}\{\lambda_n^jp_n^j+\lambda_n^{j+1}(1-p_n^{j+1})\}(S-S_{n+1}^{j+1})]
\end{aligned}
\end{equation*}
}

\frame{
\begin{equation*}
\begin{aligned}
P(S,n\Delta t)&=e^{-r\Delta t}\{\lambda_n^1(1-p_n^1)(S-S_{n+1}^1)\\
&+[\lambda_n^1p_n^1+\lambda_n^2(1-p_n^2)](S-S_{n+1}^2)\\
&+[\lambda_n^2p_n^2+\lambda_n^3(1-p_n^3)](S-S_{n+1}^3)\\
&+\ldots\\
&+[\lambda_n^{i-1}p_n^{i-1}+\lambda_n^i(1-p_n^i)](S-S_{n+1}^i)\}\\
&=e^{-r\Delta t}[\lambda_n^{i}(1-p_n^{i})(S-S_{n+1}^i)\\
&+\sum_{j=1}^{i-1}\[\lambda_n^j\{(1-p_n^j)(S-S_{n+1}^j)+P_n^j(S-S_{n+1}^{j+1})\}\\
\end{aligned}
\end{equation*}
\begin{equation*}
\begin{aligned}
P(S,n\Delta t)=e^{-r\Delta t}\{\lambda_n^{i}(1-p_n^{i})(S-S_{n+1}^i)+\sum_{j=1}^{i-1}\lambda_n^j(S_n^i-F_n^i)\}
\end{aligned}
\end{equation*}
}

\end{document}








